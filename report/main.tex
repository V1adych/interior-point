\documentclass{article}
\usepackage{graphicx}
\usepackage{amsmath}
\usepackage[T2A]{fontenc}  % For Cyrillic fonts
\usepackage[utf8]{inputenc}  % For UTF-8 encoding
\usepackage[russian]{babel}  % For Russian language support
\usepackage{listings}
\usepackage{xcolor}
\usepackage{mathtools}
\usepackage{color}
\usepackage{hyperref}

\hypersetup{
    colorlinks=true,
    linkcolor=blue,
    filecolor=blue,
    urlcolor=blue,
    citecolor=blue
}

\definecolor{dkgreen}{rgb}{0,0.6,0}
\definecolor{gray}{rgb}{0.5,0.5,0.5}
\definecolor{mauve}{rgb}{0.58,0,0.82}

\lstset{frame=tb,
    language=Python,
    aboveskip=3mm,
    belowskip=3mm,
    showstringspaces=false,
    columns=flexible,
    basicstyle={\small\ttfamily},
    numbers=left,
    numberstyle=\tiny\color{gray},
    keywordstyle=\color{blue},
    commentstyle=\color{dkgreen},
    stringstyle=\color{mauve},
    breaklines=true,
    breakatwhitespace=true,
    tabsize=3
}
\title{Report of Programming Task 2 of the course "Introduction to Optimization" - Fall 2024}
\author{Nikita Zagainov, Ilyas Galiev, Arthur Babkin, Nikita Menshikov, \\ Sergey Aitov}
\date{September 2024}

\begin{document}

\maketitle

\section{Team Information}
\noindent
\begin{itemize}
    \item Team leader: Nikita Zagainov --- 5

          Managed team work, Contributed to the algorithm implementation, Wrote report

    \item Team member 1: Ilyas Galiev --- 5

          Contributed to the algorithm implementation

    \item Team member 2: Arthur Babkin --- 5

          Contributed to the algorithm implementation, Wrote QA tests
    
          \item Team member 3: Nikita Menshikov --- 5

          Contributed to the algorithm implementation, Adapted problem from previous
          assignment

    \item Team member 4: Sergey Aitov --- 5

          Contributed to the algorithm implementation, Contributed to QA testing
\end{itemize}
\section{Link to the product}
\href{https://github.com/V1adych/interior-point}{Project source code}

\section{Programming language}
Python

\section{Linear programming problem}
We aim to maximize nutritious value of salad given constraints on cost of its
ingredients, maximum fats concentration, and weight of each individual
component

\begin{table}[h!]
    \centering
    \begin{tabular}{|p{2cm}||c|c|c|c|c|}
        \hline
        Ingredient                & Tomato & Cucumber & Bell Pepper & Lettuce Leaf & Onion \\
        \hline \hline
        Cost, rub/kg              & 130    & 100      & 155         & 85           & 50    \\
        \hline
        Nutritious value, ckal/kg & 200    & 160      & 260         & 150          & 400   \\
        \hline
        Max weight in salad, kg   & 0.6    & 0.6      & 0.6         & 0.2          & 0.05  \\
        \hline
        Fats, proportion          & 0.004  & 0.005    & 0.006       & 0.003        & 0.004 \\
        \hline
    \end{tabular}
    \caption{Ingredients and their properties}
    \label{table:ingredients}
\end{table}
\begin{itemize}
    \item Our problem is maximization problem
    \item Objective function \& constraints:
          \[
              \text{maximize } c^T x
          \]

          subject to

          \[
              Ax \leq b
          \]

          where:

          \[
              A = \begin{bmatrix}
                  130   & 100   & 155   & 85    & 50    \\
                  0.004 & 0.005 & 0.006 & 0.003 & 0.004 \\
                  1     & 0     & 0     & 0     & 0     \\
                  0     & 1     & 0     & 0     & 0     \\
                  0     & 0     & 1     & 0     & 0     \\
                  0     & 0     & 0     & 1     & 0     \\
                  0     & 0     & 0     & 0     & 1
              \end{bmatrix}
          \]

          \[
              b = \begin{bmatrix}
                  200 \\
                  1   \\
                  0.6 \\
                  0.6 \\
                  0.6 \\
                  0.2 \\
                  0.05
              \end{bmatrix}
          \]

          \[
              c = \begin{bmatrix}
                  200 \\
                  160 \\
                  260 \\
                  150 \\
                  400
              \end{bmatrix}
          \]
\end{itemize}

However, since the algorithm we implemented solves maximization problem
\[
    \text{maximize } c^T x
\]
subject to
\[
    Ax = b
\]
We manually introduce slack variables to convert inequality constraints to
equality constraints.

\section{Output \& Results}
We tested our implementation of interior point method by comparing its outputs
with \href{https://scipy.org}{scipy} implementation, and all tests show that
outputs of both methods are the same on multiple tests, including original
problem. \\ The method is applicable to our problem:
\begin{align*}
     & \text{Problem is bounded: True}          \\
     & x: [0.2115, \ 0.6, \ 0.6, \ 0.2, \ 0.05] \\
     & f: 344.3
\end{align*}

The results match with our previous simplex method implementation.

\section{Code}

\begin{lstlisting}[language=Python]
import numpy as np
from typing import Tuple


def pivot_col(tableau: np.ndarray, tol: float) -> int:
    last_row = tableau[-1, :-1]
    if np.all(last_row >= -tol):
        return -1
    return np.argmin(last_row)


def pivot_row(tableau: np.ndarray, tol: float, col: int) -> int:
    rhs = tableau[:-1, -1]
    lhs = tableau[:-1, col]
    ratios = np.full_like(rhs, np.inf)
    valid = lhs > tol
    ratios[valid] = rhs[valid] / lhs[valid]
    if np.all(ratios == np.inf):
        return -1
    return np.argmin(ratios)


def find_basic_solution(
    A: np.ndarray, b: np.ndarray, tol: float = 1e-6
) -> Tuple[bool, np.ndarray]:
    """
    Performs Phase I of the simplex method to find a basic feasible solution.

    Args:
        A: Coefficient matrix of the constraints (m x n).
        b: Right-hand side vector of the constraints (m,).
        tol: Tolerance for determining feasibility.

    Returns:
        A tuple containing:
        - A boolean indicating whether a feasible solution was found.
        - A numpy array representing the basic feasible solution (if found).
    """
    m, n = A.shape
    A_phase1 = np.hstack([A, np.eye(m)])
    c_phase1 = np.concatenate([np.zeros(n), np.ones(m)])
    B = list(range(n, n + m))

    tableau = np.hstack([A_phase1, b.reshape(-1, 1)])
    tableau = np.vstack([tableau, np.concatenate([c_phase1, [0]])])

    for i in range(m):
        tableau[-1, :] -= tableau[i, :]

    while True:
        col = np.argmin(tableau[-1, :-1])
        if tableau[-1, col] >= -tol:
            break

        ratios = []
        for i in range(m):
            if tableau[i, col] > tol:
                ratio = tableau[i, -1] / tableau[i, col]
                ratios.append((ratio, i))
        if not ratios:
            return False, None
        _, row = min(ratios)

        pivot = tableau[row, col]
        tableau[row, :] /= pivot
        for i in range(m + 1):
            if i != row:
                tableau[i, :] -= tableau[i, col] * tableau[row, :]

        B[row] = col

    basic_solution = np.zeros(n + m)
    basic_solution[B] = tableau[:m, -1]
    if np.any(basic_solution[n:] > tol):
        return False, None

    x_basic = basic_solution[:n]
    x_basic = x_basic + 2e-5  # this is to avoid zero gradients on first step
    return True, x_basic


def interior_point(
    A: np.ndarray,
    b: np.ndarray,
    c: np.ndarray,
    alpha: float = 0.5,
    tol: float = 1e-6,
    max_iters: int = 100000,
) -> Tuple[bool, np.ndarray]:
    """
    Performs the interior point method to solve a linear programming problem.

    Args:
        A: Coefficient matrix of the constraints (m x n).
        b: Right-hand side vector of the constraints (m,).
        c: Coefficient vector of the objective function to be maximized (n,).
        alpha: Step size for the Newton step.
        tol: Tolerance for determining convergence.
        max_iters: Maximum number of iterations to perform.

    Returns:
        A tuple containing:
        - A boolean indicating whether a feasible solution was found.
        - A numpy array representing the optimal solution (if found).
    """
    bounded, x = find_basic_solution(A, b)
    if not bounded:
        return False, None
    prev_x = None
    n_iters = 0

    while prev_x is None or np.linalg.norm(x - prev_x) > tol:
        n_iters += 1
        if n_iters > max_iters:
            return False, None, None

        D = x.copy()
        x_hat = x * (1 / D)
        A_hat = A * D
        c_hat = c * D
        P = np.eye(A_hat.shape[1]) - A_hat.T @ np.linalg.inv(A_hat @ A_hat.T) @ A_hat

        c_proj = P @ c_hat
        nu = np.abs(np.min(c_proj)) if np.any(c_proj < 0) else 1
        x_hat = x_hat + (alpha / nu) * c_proj
        prev_x = x.copy()
        x = D * x_hat

    return True, x, c @ x


def main():
    np.set_printoptions(precision=4, suppress=True)

    A = np.array(
        [
            [130, 100, 155, 85, 50],
            [0.004, 0.005, 0.006, 0.003, 0.004],
            [1, 0, 0, 0, 0],
            [0, 1, 0, 0, 0],
            [0, 0, 1, 0, 0],
            [0, 0, 0, 1, 0],
            [0, 0, 0, 0, 1],
        ],
        dtype=np.float32,
    )
    b = np.array([200, 0.01, 0.6, 0.6, 0.6, 0.2, 0.05], dtype=np.float32)
    c = np.array([200, 160, 260, 150, 400], dtype=np.float32)

    # since we are maximizing c.T @ x s. t. A @ x <= b, we need to introduce slack variables:
    A_slack = np.hstack([A, np.eye(A.shape[0])])
    c_slack = np.concatenate([c, np.zeros(A.shape[0])])

    feasible, x, f = interior_point(A_slack, b, c_slack)
    if feasible:
        print(f"Optimal x: {x[: c.shape[0]]}, optimal value: {f}")
    else:
        print("No feasible solution found.")


if __name__ == "__main__":
    main()     
\end{lstlisting}
\end{document}
